\documentclass[dvips]{article}
\usepackage{url}
\usepackage{fullpage}
\usepackage{hevea}
\usepackage[pdftex]{graphicx}

\begin{document}

\title{Biopython
Tutorial and Cookbook}
\author{Jeff Chang and Brad Chapman}
\date{Last
Update--1 July 00}

\section{Introduction}

\subsection{What is Biopython?}

\subsection{Obtaining and Installing Biopython}

I think it's nice to have all of the documentation on everything
together, but we can take it out if you think it'll make it too big.

Tips and Tricks

\subsection{FAQ}



\section{Quick Start -- 
What can you do with Biopython?}

This is going to be the meat of the tutorial.  We should run people
through an example of things, much like in the ACM SIGBIO paper.  We
should not get too technical.  It's intended audience is for new users
of biopython.  That is, we should give just enough detail for people
to start hacking on their own.

We can assume: internet connection, knowledge of Python, biopython
installed.



\section{Cookbook -- Cool things to do with it}

\subsection{BLAST}

\subsection{SWISS-PROT}

\subsubsection{PubMed}

\subsection{Classification}

\subsection{BioCorba}

\subsection{Miscellaneous}



\section{Advanced}

\subsection{Sequence Class}

\subsection{Regression Testing Framework}

\subsection{Parser Design}



\section{Biocorba}

Just a quick introduction and a pointer to the
Biocorba documentation.


\section{Where to go from here -- contributing to
Biopython}

\subsection{Bug Reports + Feature
Requests}

\subsection{Contributing Code}

\end{document}
