\documentclass[dvips]{article}
\usepackage{url}
\usepackage{fullpage}
\usepackage{hevea}
\usepackage[pdftex]{graphicx}

\begin{document}

\title{Biopython
Tutorial and Cookbook}
\author{Jeff Chang and Brad Chapman}
\date{Last Update--4 August 00}

\section{Introduction}

\subsection{What is Biopython?}

\subsection{Obtaining and Installing Biopython}

Biopython's internet home is at, naturally enough,  \ahrefurl{\url{http://www.biopython.org}}. This is the home of all things biopython, so it is the best place to start looking around if you are interested. When you feel ready to dive in and start working with the code, you have two choices:

\begin{enumerate}

\item Release code -- We made available both stable and developer's releases on the download page (\ahrefurl{\url{http://www.biopython.org/Download/}}). The stable releases are likely to be more well tested, while the development releases are closer to what is in CVS, and so will probably have more features.

\item CVS -- The current working copy of the Biopython sources is always available via CVS (Concurrent Versions Systems -- \ahrefurl{\url{http://www.cvshome.org/}}). Concise instructions for accessing this copy are available at \ahrefurl{\url{http://cvs.biopython.org}}.

\end{enumerate}

No matter which way you choose, installing Biopython will require the following easy steps.

\subsubsection{Installation}

Biopython uses Distutils, which is the new standard python installation package. Copies are available at \ahrefurl{\url{http://www.python.org/sigs/distutils-sig/download.html}} and Distutils will also come standard with Python 1.6 and future versions (the version that is shipped with alpha versions of 1.6 should work just fine). Distutils will make installation a snap, as you will see in a second.


Now that we've got what we need let's get into the installation: (Warning: These instructions are strongly biased to UNIX type machines. If you are installing on Windows or Mac or some brand of UNIX for which these instructions are not very helpful, please let us know how we can make them better!)

\begin{enumerate}

\item First you need to unpack the distribution. If you got the CVS version, you are all set to go and can skip on ahead. Otherwise, you'll need to unpack it. On UN*X machines, a tar.gz package is provided, which you can unpack with \verb|tar -xzvpf biopython-X.X.tar.gz|. A zip file is also provided for other platforms. XXX Add stuff about rpms once we get them working.

\item Now that everything is unpacked, move into the \verb| biopython*| directory (this will just be \verb|biopython| for CVS users, and will be \verb|biopython-X.X| for those using a packaged download). 

\item Now you are ready for your one step install -- \verb|python setup.py install|. This performs the default install, and will put Biopython into the \verb|site-packages| directory of your python library tree (on my machine this is \verb|/usr/local/python1.5/site-packages|). You will have to have permissions to write to this directory. 

\begin{enumerate}

\item This install requires that you have the python source available. You can check this by looking for \verb|Python.h| and \verb|config.h| in some place like \verb|/usr/local/include/python1.5|.

\item The distutils setup process allows you to do some customization of your install so you don't have to stick everything in the default location (in case you don't have write permissions there, or just want to test Biopython out). You have quite a few choices, which are covered in detail in the distutils installation manual (http://www.python.org/sigs/distutils-sig/doc/inst/inst.html), specifically in the Alternative installation section.

\end{enumerate}

\item That's it! Biopython is installed. Wasn't that easy? Now let's check and make sure it worked properly.

\end{enumerate}

\subsubsection{Making sure it worked}

First, we'll just do a quick test to make sure python. The most important thing is that python can find the biopython installation. Biopython all installs into a top level \verb|Bio| directory, and you want to make sure this is on your\verb| $PYTHONPATH| on UNIX (hmmm, what do you do on other platforms?). If you used the default install, this shouldn't be a problem, but if not, you'll need to set the \verb|PYTHONPATH| with something like \verb|export PYTHONPATH = $PYTHONPATH':/directory/where/you/put/Biopython'|. Now that we think we are ready, fire up your python interpreter and follow allong with the following quick script:

\begin{verbatim}

[chapmanb@taxus chapmanb]$ python
Python 1.6a2 (#1, Jul 31 2000, 09:04:26)  [GCC 2.95.2 19991024 (release/franzo)] on linux2
Copyright 1991-1995 Stichting Mathematisch Centrum, Amsterdam
>>> from Bio.Seq import Seq
>>> new_seq = Seq('GATC') 
>>> new_seq[0:2]
Seq('GA', Alphabet())

\end{verbatim}

If this worked properly, then it looks like Biopython is in a happy place where python can find it, so now you might want to do some more rigorous tests. The \verb|Tests| directory inside the distribution contains a number of tests you can run to make sure all of the different parts of biopython are working. These should all work just by running \verb|python test_WhateverTheTestIs.py|. You can also do a run of all of the tests py typing \verb|python br_regrtest.py| in the Tests directory:

\begin{verbatim}
[chapmanb@taxus Tests]$ python br_regrtest.py
test_Enzyme
test_Fasta
test_Fasta2
test_File
....
\end{verbatim}


Well, now you've gotten Biopython installed and running, you are probably ready to get working with it, so continue reading...

\subsection{FAQ}



\section{Quick Start -- What can you do with Biopython?}

This section is designed to get you started quickly with Biopython, and to give a general overview of what is available and how to use it. All of the examples in this section assume that you have some general working knowledge of python, and that you have successfully installed biopython on your system. If you think you need to brush up on your python, the main python web site provides quick a bit of free documentation to get started with at \ahrefurl{\url{http://www.python.org/doc/}}. 


Since much biological work on the computer involves connecting with databases on the internet, some of the examples will also require a working internet connection in order to run. 


Now that that is all out of the way, let's get into what we can do with Biopython.

\subsection{General overview of what Biopython provides}

As mentioned in the introduction, Biopython is a set of libraries to provide the ability to deal with ''things'' of interest to biologists working on the computer. In general this means that you will need to have at least some programming experience (in python, of course!) or at least an interest in learning to program. Biopython's job is to make your job easier as a programmer by supplying reusable libraries so that you can focus on answering your specific question of interest, instead of focusing on the internals of parsing a particular file format (of course, if you want to help by writing a parser that doesn't exist and contributing it to Biopython, please go ahead!). So Biopython's job is to make you happy!


One thing to note about Biopython is that it often provides multiple ways of ``doing the same thing.'' To me, this can be frustrating since I  often way to just know the one right way to do something. However, this is also a real benefit because it gives you lots of flexibility and control over the libraries. The tutorial helps to show you the common or easy ways to do things so that you can just make things work. To learn more about the alternative possibilities, look into the Cookbook section (which tells you some cools tricks and tips) and the Advanced section (which provides you with as much detail as you'd ever want to know!). 

\subsection{Working with sequences}

Disputedly (of course!), the central object in bioinformatics is the sequence. Thus, we'll start with the Biopython mechanisms for dealing with sequences. When I think of a sequence the first thing that pops into my mind is a string of letters:\verb| 'AGTACACTGGT'| which seems natural since this is the most common way that sequences are seen in biological file formats.  However, a simple string of letters by itself is also very uninformative -- is it a DNA or  protein sequence (okay, a protein with a lot of Alanines, Glycines, Cysteines and Threonines!), what type of organism did it come from, what is so interesting about it, and so on. The challenge in designing a sequence interface is to pick a representation that is informative enough to take into account the more complex information, yet is as lightweight and easy to work with as just a simple sequence.


The approach taken in the Biopython sequence class is to utilize a class that holds more complex information, yet can be manipulated as if it were a simple string. This is accomplished by utliziing operator overloading to make manipulating a sequence object feel like manipulating a python string. The sequence class, referred to simply as Seq,  is defined in the file \verb|Bio/Seq.py.| Let's look at the Seq class deeper to see what it has to offer.


A biopython Seq object has two important attributes:

\begin{enumerate}

\item \verb|data| -- as the name implies, this is the actual sequence data string of the sequence.

\item \verb|alphabet| -- an object describing what the individual characters making up the string ``mean'' and how they should be interpreted.

\end{enumerate}

Clearly the alphabet object is the important thing that is making the Seq object more than just a string. The currently available alphabets for Biopython are defined in the \verb|Bio/Alphabet| module. We'll use the IUPAC alphabets \ahrefurl ({\url{http://www.chem.qmw.ac.uk/iupac/}}) here to deal with some of our favorite objects: DNA, RNA and Proteins.  


\verb|Bio/Alphabet/IUPAC.py| provides basic definitions for proteins, DNA and RNA, but additionally provides the abilitiy to extend and cutomize the basic definitions. For instance, for proteins, there is a basic IUPACProtein class, but there is an additional ExtendedIUPACProtein class providing for the additional elements ``Asx'' (asparagine or aspartic acid), ``Sec'' (selenocysteine), and ``Glx'' (glutamine or glutamic acid). For DNA you've got choices of IUPACUnambiguousDNA, which provides for just the basic letters, IUPACAmbiguousDNA (which provides for ambiguity letters for every possible situation) and ExtenedIUPACDNA, which allows letters for modified bases. Similarly, RNA can be represented by IUPACAmbigousRNA or IUPACUnambigousRNA.


The advantages of having an alphabet class are two fold. First, this gives an idea of the type of information the \verb|data| object contains. Secondly, this provides a means of contraining the information you have in the data object, as a means of type checking.


Now that we know what we are dealing with, let's look at how to utilize this class to do interesting work.


First, create a Sequence object from a string of information we've got. We'll create an unambiougous DNA object:

\begin{verbatim}
>>> from Bio.Alphabet import IUPAC
>>> my_alpha = IUPAC.unambiguous_dna
>>> from Bio.Seq import Seq
>>> my_seq = Seq('GATCGATGGGCCTATATAGGATCGAAAATCGC', my_alpha)
>>> print my_seq
Seq('GATCGATGGGCCTATATAGGATCGAAAATCGC', IUPACUnambiguousDNA())
\end{verbatim}


Even though this is a sequence object, we can deal with it in some ways as if it were a normal python string. For instance, let's get a slice of the sequence.

\begin{verbatim}
>>> my_seq[4:12]
Seq('GATGGGCC', IUPACUnambiguousDNA())
\end{verbatim}

Two things are interesting to note. First, this follows the normal conventions for python sequences.  So the first element of the sequence is 0 (which is normal for computer science, but not so normal for biology). When you do a slice the first item is included (ie. 4 in this case) and the last is excluded (12 in this case), which is the way things work in python, but of course not necessarily the way everyone in the world would expect. The main goal is to stay consistent with what python does. The second thing to notice is that the slice is performed on the sequence data string, but the new object produced retains the alphabet information from the original Seq object.


You can treat the Seq object like the string in many ways:

\begin{verbatim}
>>> len(my_seq)
32
>>> new_seq = my_seq[0:5]
>>> print new_seq
Seq('GATCG', IUPACUnambiguousDNA())
>>> my_seq + new_seq
Seq('GATCGATGGGCCTATATAGGATCGAAAATCGCGATCG', IUPACUnambiguousDNA())
>>> my_seq[5]
'A'
>>> my_seq == new_seq
0
\end{verbatim}

In all of the operations, the alphabet property is maintained. This is very useful in case you accidentally end up trying to do something weird like add a protein sequence and a DNA sequence:

\begin{verbatim}
>>> protein_seq = Seq('EVRNAK', IUPAC.protein)
>>> dna_seq = Seq('ACGT', IUPAC.unambiguous_dna)
>>> protein_seq + dna_seq
Traceback (most recent call last):
  File "<stdin>", line 1, in ?
  File "/usr/local/lib/python1.6/site-packages/Bio/Seq.py", line 42, in __add__
    raise TypeError, ("incompatable alphabets", str(self.alphabet),
TypeError: ('incompatable alphabets', 'IUPACProtein()', 'IUPACUnambiguousDNA()')
\end{verbatim}


And if you are really just need the string to insert into something, this is very easy to extract:

\begin{verbatim}
>>> my_seq.tostring()
'GATCGATGGGCCTATATAGGATCGAAAATCGC'
\end{verbatim} 

The sequence object is not mutable by default, since in many biological applications you want to ensure you are not changing your data:

\begin{verbatim}
>>> my_seq[5] = 'G'
Traceback (most recent call last):
  File "<stdin>", line 1, in ?
AttributeError: 'Seq' instance has no attribute '__setitem__'
\end{verbatim}

However, you can convert it into a mutable sequence and do pretty much anything you want with it:

\begin{verbatim}
>>> mutable_seq = my_seq.tomutable()
>>> print mutable_seq
MutableSeq('GATCGATGGGCCTATATAGGATCGAAAATCGC', IUPACUnambiguousDNA())
>>> mutable_seq[5] = 'T'
>>> print mutable_seq
MutableSeq('GATCGTTGGGCCTATATAGGATCGAAAATCGC', IUPACUnambiguousDNA())
>>> mutable_seq.remove('T')
>>> print mutable_seq
MutableSeq('GACGTTGGGCCTATATAGGATCGAAAATCGC', IUPACUnambiguousDNA())
>>> mutable_seq.reverse()
>>> print mutable_seq
MutableSeq('CGCTAAAAGCTAGGATATATCCGGGTTGCAG', IUPACUnambiguousDNA())
\end{verbatim}

Now that the nature of the sequence object makes some sense, the next thing to look at is what kind of things we can do with a sequence. The \verb|Bio/Tools| directory contains two useful modules to transcribe and translate a sequence object. These tools work based on the alphabet of the sequence. For instance, let's supposed we want to transcribe our \verb|my_seq| object. Remember that this contains an unambiguous alphabet, so to transcribe we would do the following:

\begin{verbatim}
>>> from Bio.Tools import Transcribe
>>> transcriber = Transcribe.unambiguous_transcriber
>>> my_rna_seq = transcriber.transcribe(my_seq)
>>> print my_rna_seq
Seq('GAUCGAUGGGCCUAUAUAGGAUCGAAAAUCGC', IUPACUnambiguousRNA())
\end{verbatim}

The alphabet of the new RNA Seq object is created for free for you, so again, dealing with a Seq object is no more difficult then dealing with a simple string.


You can also reverse transcribe RNA sequences:

\begin{verbatim}
>>> transcriber.back_transcribe(my_rna_seq)
Seq('GATCGATGGGCCTATATAGGATCGAAAATCGC', IUPACUnambiguousDNA())
\end{verbatim}


To translate our DNA object we have quite a few choices. First, we can use any number of translation tables depending on what we know about our DNA sequence. The translation tables available in biopython were taken from information at \ahrefurl{\url{ftp://ncbi.nlm.nih.gov/entrez/misc/data/gc.prt}}. So, you have tons of choices to pick from. For this, let's just focus on two choices: the Standard translation table, and the Translation table for Vertebrate Mitochondriall DNA. These tables are labelled with id numbers 1 and 2, respectively. Now that we know what tables we are looking to get, we're all set to perform a basic translation. First, we need to get our translators that use these tables. Since we are still dealing with our unambiguous DNA object, we want to fetch translators that take this into account:

\begin{verbatim}
>>> from Bio.Tools import Translate
>>> standard_translator = Translate.unambiguous_dna_by_id[1] 
>>> mito_translator = Translate.unambiguous_dna_by_id[2]
\end{verbatim}

Once we've got the proper translators, it's time to go ahead and translate a sequence:

\begin{verbatim}
>>> my_seq = Seq('GCCATTGTAATGGGCCGCTGAAAGGGTGCCCGA', IUPAC.unambiguous_dna)
>>> standard_translator.translate(my_seq)
Seq('AIVMGR*KGAR', IUPACProtein())
>>> mito_translator.translate(my_seq)
Seq('AIVMGRWKGAR', IUPACProtein())
\end{verbatim}

Notice that the default translation will just go ahead and proceed blindly through a stop codon. If you are aware that you are translating some kind of open reading frame and want to just see everything up until the stop codon, this can be easily done with the \verb|translate_to_stop| function:

\begin{verbatim}
>>> standard_translator.translate_to_stop(my_seq)
Seq('AIVMGR', IUPACProtein())
\end{verbatim}

Similar to the transcriber, it is also possible to reverse translate a protein into a DNA sequence:

\begin{verbatim}
>>> my_protein = Seq('AVMGRWKGGRAAG', IUPAC.protein)
>>> standard_translator.back_translate(my_protein)
Seq('GCTGTTATGGGTCGTTGGAAGGGTGGTCGTGCTGCTGGT', IUPACUnambiguousDNA())
\end{verbatim}

This covers the basic features and uses of the Biopython sequence class. There is a more detailed description of the design ideas behind the sequence class in the Advanced section of this tutorial. This class is still under development and comments on the design and use are, of course, very welcome. Now that you've got some idea of what it is like to interact with the Biopython libraries, it's time to delve into the fun, fun world of dealing with biological file formats!

\subsection{A usage example}

Before we jump right into parsers and everything else to do with Biopython, let's set up an example to motivate everything we do and make life more interesting. After all, if there wasn't any biology in this tutorial, why would you want you read it?


Since I love plants, I think we're just going to have to have a plant based example (sorry to all the fans of other organisms out there!).  Having just completed a recent trip to our local greenhouse, we've suddenly developed an incredible obsession with Lady Slipper Orchids (if you wonder why take a look at \ahrefurl{\url{http://www.millicentorchids.com/greenhouse/images/papesq01.jpg}}. If that doesn't convince you, you can look at all of the available photos at  \ahrefurl{\url{http://www.millicentorchids.com/greenhouse/indexphoto.htm}}).  Of course, orchids are not only beautiful to look at, they are also extermely interesting for people studying evolution and systematics. So we're thinking about writing a little proposal to do a molecular study of Lady Slipper evolution and would like to see what kind of research has already been done and how we can add to that.


After a little bit of reading up we discover that the Lady Slipper Orchids are in the Orchidaceae family and the Cypripedioideae sub-family and are made up of 5 genera:  \emph{Cypripedium}, \emph{Paphiopedilum}, \emph{Phragmipedium}, \emph{Selenipedium} and \emph{Mexipedium}. That gives us enough information to get started delving for more information. So, let's look at how biopython tools can help us.

\subsection{Parsing biological file formats}

A large part of much bioinformatics work involves dealing with the many types of file formats designed to hold biological data. These files are loaded with interesting biological data, and a special challenge is parsing these files into a format so that you can manipulate them with some kind of programming language. However the task of parsing these files can be frustrated by the fact that the formats can change quite regularly, and that formats may contain small subleties which can break even the most well designed parsers. 

\subsubsection{General parser design}

The biopython solution to these problem is to develop a structured parser framework that is applicable to all of the parsers. The advantages are two fold. First, this allows code reuse between parsers (see \verb|Bio/ParserSupport.py|). Second, this provides a similar framework for all the parsers so that it is relatively easy to jump into the internals of a parser and figure out problems you might be having. 


All of the parsers have two components:

\begin{enumerate}

\item Scanner - The part of the parser that actually does the work or going through the file and extracting useful information. This useful information is converted into Events.

\item Consumer - The consumer does the job of processing the useful information and spitting it out ina format that the programmer can use. The consumer does this by recieving the events created by the scanner.

\end{enumerate}

So, the parser design is event oriented. The general idea is that a scanner will go through and produce events for every item that might be of interest in a file. For instance, let's say we've got the following FASTA formatted entry (edited to fit the page):


\begin{verbatim}
>gi|6273290|gb|AF191664.1|AF191664 Opuntia clavata rpl16 gene; chloroplast gene for...
TATACATTAAAGGAGGGGGATGCGGATAAATGGAAAGGCGAAAGAAAGAAAAAAATGAA
TCTAAATGATATAGGATTCCACTATGTAAGGTCTTTGAATCATATCATAAAAGACAATGTAAT
AAA...
\end{verbatim}

As a scanner moved through a file containing this entry, it would create the following events:
\\
\\
\begin{tabular}{cc}
Event Name & Entry input \\
\hline
begin\_sequence & (as soon as we notice we've got a new \verb|>|) \\
title & \verb+>gi|6273290|gb|AF191664.1|AF191664 Opuntia clavata...+ \\
sequence & \verb|TATACATTAAAGGAGGGGGATGCGGAT...| \\
sequence & \verb|TCTAAATGATATAGGATTCCACTATGTAA...| \\
sequence & \verb|AAA...| (and so on -- you've got the idea!) \\
end\_sequence & (as soon as we reach a blank line after the sequence data) \\

\end{tabular}
\\
\\
So, events are being produced. Big deal, unless we are able to capture these events and do something interesting with them! This is where the consumer comes in. The consumer needs to do two things:

\begin{enumerate}

\item Register itself with the the scanner to let it know that it wants to recieve those events that are being generated.

\item Do something with the events (and the information associated with them).

\end{enumerate}

An example should make it more clear how this works.

\subsubsection{Writing your own consumer}

Now it's time to understand our parser framework, and also start looking at our friends, the lady slipper orchids. To start our search, let's just take a look through the nucleotide databases at NCBI, using an Entrez search (\ahrefurl{\url{lhttp://www.ncbi.nlm.nih.gov:80/entrez/query.fcgi?db=Nucleotide}}) for everything mentioning the text Cypripedioideae (this is the subfamily of lady slipper orchids). This search gives us 94 hits, which we save into a FASTA formatted text file.


Now, let's try and use a parser to summarize the type of information we got. To take a simple example, why don't we go through and grab all of the organism names mentioned, to see how many different species of lady slipper orchids are represented in this data.

\subsection{Connecting with biological databases}

\subsection{Other nifty things}

\subsection{What to do next}

\section{Cookbook -- Cool things to do with it}

\subsection{BLAST}

\subsubsection{Running BLAST over the internet}

\subsubsection{Running BLAST locally}

\subsubsection{Parsing BLAST output}

\subsubsection{Parsing a file full of BLAST runs}

\subsection{SWISS-PROT}

\subsubsection{Retrieving a SWISS-PROT record}

\subsection{PubMed}

\subsubsection{Sending a query to PubMed}

\subsubsection{Retrieving a PubMed record}

\subsection{Classification}

\subsection{BioCorba}

\subsection{Miscellaneous}

\subsubsection{Translating a DNA sequence to Protein}

\section{Advanced}

\subsection{Sequence Class}

\subsection{Regression Testing Framework}

\subsection{Parser Design}


Design Overview
---------------


Parsers are built around an event-oriented design that includes
Scanner and Consumer objects.


Scanners take input from a data source and analyze it line by line,
sending off an event whenever it recognizes some information in the
data.  For example, if the data includes information about an organism
name, the scanner may generate an \verb|organism_name| event whenever it
encounters a line containing the name.


Consumers are objects that receive the events generated by Scanners.
Following the previous example, the consumer receives the
\verb|organism_name| event, and the processes it in whatever manner
necessary in the current application.



Events
------


There are two types of events: info events that tag the location of
information within a data stream, and section events that mark
sections within a stream.  Info events are associated with specific
lines within the data, while section events are not.


Section event names must be in the format \verb|start_EVENTNAME| and
\verb|end_EVENTNAME| where \verb|EVENTNAME| is the name of the event.


For example, a FASTA-formatted sequence scanner may generate the
following events:
\begin{verbatim}
EVENT NAME      ORIGINAL INPUT
begin_sequence  
title           >gi|132871|sp|P19947|RL30_BACSU 50S RIBOSOMAL PROTEIN L30 (BL27
sequence        MAKLEITLKRSVIGRPEDQRVTVRTLGLKKTNQTVVHEDNAAIRGMINKVSHLVSVKEQ
end_sequence
begin_sequence
title           >gi|132679|sp|P19946|RL15_BACSU 50S RIBOSOMAL PROTEIN L15
sequence        MKLHELKPSEGSRKTRNRVGRGIGSGNGKTAGKGHKGQNARSGGGVRPGFEGGQMPLFQRLPK
sequence        RKEYAVVNLDKLNGFAEGTEVTPELLLETGVISKLNAGVKILGNGKLEKKLTVKANKFSASAK
sequence        GTAEVI
end_sequence
[...]
\end{verbatim}

(I cut the lines shorter so they'd look nicer in my editor).


The FASTA scanner generated the following events: \verb|title|, \verb|sequence|,
\verb|begin_sequence|, and \verb|end_sequence|.  Note that the \verb|begin_sequence|
and \verb|end_sequence| events are not associated with any line in the
original input.  They are used to delineate separate sequences within
the file.


The events a scanner can send must be specifically defined for each
data format.



'noevent' EVENT
-----------------


A data file can contain lines that have no meaningful information,
such as blank lines.  By convention, a scanner should generate the
"noevent" event for these lines.




Scanners
--------


\begin{verbatim}
class Scanner:
    def feed(self, handle, consumer):
        # Implementation
\end{verbatim}


Scanners should implement a method named 'feed' that takes a file
handle and a consumer.  The scanner should read data from the file
handle and generate appropriate events for the consumer.



Consumers
---------

\begin{verbatim}
class Consumer:
    # event handlers
\end{verbatim}


Consumers contain methods that handle events.  The name of the method
is the event that it handles.  Info events are passed the line of the
data containing the information, and section events are passed
nothing.


You are free to ignore events that are not interesting for your
application.  You should just not implement methods for those events.


All consumers should be derived from the base Consumer class.


An example:


\begin{verbatim}
class FASTAConsumer(Consumer):
    def title(self, line):
        # do something with the title
    def sequence(self, line):
        # do something with the sequence
    def begin_sequence(self):
        # a new sequence starts
    def end_sequence(self):
        # a sequence ends
\end{verbatim}


\subsubsection{BLAST}

BLAST Scanners produce the following events:

\begin{verbatim}
header
    version
    reference
    query_info
    database_info

descriptions
    round                         psi blast
    model_sequences               psi blast
    nonmodel_sequences            psi blast
    converged                     psi blast
    description
    no_hits

alignment
    multalign                     master-slave
    title                         pairwise
    length                        pairwise
  hsp
    score                         pairwise
    identities                    pairwise
    strand                        pairwise, blastn
    frame                         pairwise, blastx, tblastn, tblastx
    query                         pairwise
    align                         pairwise
    sbjct                         pairwise

database_report
    database
    posted_date
    num_letters_in_database
    num_sequences_in_database
    num_letters_searched          RESERVED.  Currently unused.  I've never
    num_sequences_searched        RESERVED.  seen it, but it's in blastool.c..
    ka_params
    gapped                        not blastp
    ka_params_gap                 gapped mode (not tblastx)

parameters
    matrix
    gap_penalties                 gapped mode (not tblastx)
    num_hits                      
    num_sequences                 
    num_extends                   
    num_good_extends              
    num_seqs_better_e
    hsps_no_gap                   gapped (not tblastx) and not blastn
    hsps_prelim_gapped            gapped (not tblastx) and not blastn
    hsps_prelim_gap_attempted     gapped (not tblastx) and not blastn
    hsps_gapped                   gapped (not tblastx) and not blastn
    query_length
    database_length
    effective_hsp_length
    effective_query_length
    effective_database_length
    effective_search_space
    effective_search_space_used
    frameshift                    blastx or tblastn or tblastx
    threshold
    window_size
    dropoff_1st_pass
    gap_x_dropoff
    gap_x_dropoff_final           gapped (not tblastx) and not blastn
    gap_trigger
    blast_cutoff
\end{verbatim}

\subsubsection{Enzyme}
The Enzyme Scanner produces the following events:
\begin{verbatim}
record
    identification
    description
    alternate_name
    catalytic_activity
    cofactor
    comment
    disease
    prosite_reference
    databank_reference
    terminator
\end{verbatim}

\subsubsection{Fasta}
The Fasta Scanner produces the following events:
\begin{verbatim}
sequence
    title
    sequence
\end{verbatim}


\subsubsection{Fasta}
The Fasta Scanner produces the following events:
\begin{verbatim}
sequence
    title
    sequence
\end{verbatim}


\subsubsection{Medline}
The Online Services Reference Manual documents the MEDLINE format at:
\ahrefurl{\url{http://www.nlm.nih.gov/pubs/osrm_nlm.html}}

The Medline scanner produces the following events:
\begin{verbatim}
record
    undefined
    abstract_author
    abstract
    address
    author
    call_number
    comments
    class_update_date
    country
    entry_date
    publication_date
    english_abstract
    entry_month
    gene_symbol
    identification
    issue_part_supplement
    issn
    journal_title_code
    language
    special_list
    last_revision_date
    mesh_heading
    mesh_tree_number
    major_revision_date
    no_author
    substance_name
    pagination
    personal_name_as_subject
    publication_type
    number_of_references
    cas_registry_number
    record_originator
    journal_subset
    subheadings
    secondary_source_id
    source
    title_abbreviation
    title
    transliterated_title
    unique_identifier
    volume_issue
    year
    pubmed_id
\end{verbatim}    

undefined is a special event that is called for every line with a
qualifier not defined in the specification.


\subsubsection{Prosite}
The Prosite scanner produces the following events:
\begin{verbatim}
copyrights
    copyright
record
    identification
    accession
    date
    description
    pattern
    matrix
    rule
    numerical_results
    comment
    database_reference
    pdb_reference
    documentation
    terminator
\end{verbatim}

The PRODOC scanner produces the following events:
\begin{verbatim}
record
    accession
    prosite_reference
    text
    reference
\end{verbatim}


\subsubsection{SWISS-PROT}
The SProt Scanner produces the following events:
\begin{verbatim}
record
    identification
    accession
    date
    description
    gene_name
    organism_species
    organelle
    organism_classification
    reference_number
    reference_position
    reference_comment
    reference_cross_reference
    reference_author
    reference_title
    reference_location
    comment
    database_cross_reference
    keyword
    feature_table
    sequence_header
    sequence_data
    terminator
\end{verbatim}


The KeyWList scanner produces the following events:
\begin{verbatim}
header
keywords
    keyword
footer
    copyright
\end{verbatim}

\section{Biocorba}

Just a quick introduction and a pointer to the
Biocorba documentation.


\section{Where to go from here -- contributing to
Biopython}

\subsection{Bug Reports + Feature
Requests}

\subsection{Contributing Code}

\end{document}
